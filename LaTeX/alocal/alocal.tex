\documentclass[10pt,a4paper]{article}

\usepackage{hyperref}
\usepackage{listings}
\usepackage{xcolor}
\usepackage{metalogo}

\usepackage{arabtex}    % automatically includes alocal.sty

\newcommand{\sty}[1]{\textsf{#1.sty}}

\newcommand{\exts}[1]{\lstinline|<#1>|}

\newcommand{\orth}[1]{{\showfalse\arabtrue\transfalse\<#1>}}
\newcommand{\phon}[1]{{\showfalse\arabfalse\transtrue\<#1>\/}}

\newcommand{\expl}[1]{\showtrue\arabtrue\transtrue\<#1>}

\lstset{language=[LaTeX]TeX,
        basicstyle=\renewcommand{\baselinestretch}{0.95}\ttfamily,
        flexiblecolumns=true,
        keepspaces=true,
        texcs={nodiacritics,noshadda,novowels,wasl@symb,relax},
        texcsstyle=\color{purple},
        commentstyle={},
        emph={ii,uu,UW,uW,uwA,uuW,aa,H,czech},
        emphstyle=\color{red}}


\title{\sty{alocal} --- Local Extensions of \ArabTeX}

\author{Otakar Smr\v{z}\quad\texttt{<otakar-smrz@users.sf.net>}}


\begin{document}

\maketitle

\thispagestyle{empty}

\noindent
This package implements some of the notational extensions of Encode
Arabic \url{http://github.com/otakar-smrz/encode-arabic/} to work in
\ArabTeX.

The \sty{alocal} package is invoked by \ArabTeX{} as the file
containing local extensions. Either overwrite your copy of this file
in the \ArabTeX{} distribution, or enforce in some other way the
redefinitions that our version of \sty{alocal} consists of. Please
report to us any bugs or incompatibility issues.

\paragraph{Defective writing of \protect\lstset{moredelim=[is][\color{red}]{-}{-}}
                                \protect\lstinline|<-_I->|}
enables easy encoding of \expl{m_I'aT} and its derivatives, just like
\exts{_U} works for \raisebox{0pt}[10pt]{\expl{'_Ul_a'ika}}.

\medskip

\begin{tabular}{l}
\<m_I'aTuN bi-al-m_I'aTi>               \\
\<mi'aTuN bi-al-m_I'aTi>                \\
\<mi'aTuN bi-al-mi'aTi>                 \\
\<\fullvocalize m"_I'aTuN bi-al-mi'aTi> \\
\<\novocalize m"_I'aTuN bi-al-mi'aTi>   \\
\<m_I'atAni wa-m_I'AtuN>                \\
\end{tabular}

\paragraph{Doubled vowels \protect\exts{ii} and \protect\exts{uu}}
are set equivalent to \exts{I}, resp.~\exts{U}, in all contexts. This
is similar to \exts{aa} becoming \exts{A}. The distinction between
\exts{I} versus \exts{iy} and \exts{U} versus \exts{uw} is preserved.

\medskip

\begin{tabular}{p{.5\linewidth}l}
\<zuuwaarii>    &   \<zUwArI>   \\
\<zuwwaariy>                    \\
\end{tabular}

\paragraph{Silent \protect\phon{'alif} spelled as \protect\exts{UW}
           or \protect\exts{uW}}
compares more easily with the other \linebreak \exts{aW} variant of
the verbal ending. Writing \exts{uuW} is equivalent to \exts{UW}.

\medskip

\begin{tabular}{p{.5\linewidth}l}
\<katabUW>      &   \<katabUA>  \\
\<katabuW>      &   \<ramaW>    \\
\<katabuuW>     &   \<ramaWA>   \\
\end{tabular}

\bigskip

Along with that, we have fixed the \ArabTeX's bug in equating the
\exts{uwA} sequence with \exts{UA}, thus printing \orth{BUA} instead
of \orth{BuwA}. Now, both look different.

\medskip

\begin{tabular}{p{.5\linewidth}l}
\<muwA.tin>     &   \<muwA.salaT>   \\
\end{tabular}

\paragraph{Vocalization degree}
is extended with the \lstinline|\nodiacritics| option, which suppresses the
printing of \phon{^sadda} and all other diacritics. This option has an
obvious \lstinline|\noshadda| synonym, while \lstinline|\novocalize| can also be
invoked as \lstinline|\novowels|.

\paragraph{Transcription of \protect\phon{wa.sla}}
can be controlled and customized by redefining the \lstinline|\wasl@symb|
command. If it is set to \lstinline|\relax|, no \phon{wa.sla} is typeset
and the original auxiliary vowel is not dropped. The default setting
is \lstinline|\gdef\wasl@symb{'}|.

\paragraph{Czech-style transcription}
has been defined as an experimental modification of the
standard. Enforce it with \lstinline|\settrans{czech}|.

\bigskip

\settrans{czech}
\begin{tabular}{l}
\lstinline|\settrans{czech}|    \\[6pt]
\<yUladu ^gamI`u an-nAsi>       \\
\<wa-qad wuhibUW `aqlaN>        \\
\<bi-rU.hi al-'i_hA'i>          \\
\end{tabular}

\medskip

\settrans{standard}
\begin{tabular}{l}
\lstinline|\settrans{standard}| \\[6pt]
\<yUladu ^gamI`u an-nAsi>       \\
\<wa-qad wuhibUW `aqlaN>        \\
\<bi-rU.hi al-'i_hA'i>          \\
\end{tabular}

\paragraph{Fixing the \protect\phon{_h} accent}
is noticable with fonts other than Computer Modern.

\bigskip

\hrule

\paragraph{Consistency of \protect\phon{wa.sla}}
needs to be improved so that vowel tying does not apply across
paragraph boundaries.

% \paragraph{\protect\phon{iqra'na}}
% \begin{RLtext}iqra'na\end{RLtext}
% \begin{RLtext}iqra'na\end{RLtext}

\medskip

\begin{RLtext}
iqra'
iqra'I
iqra'iy
iqra'A
iqra'UW
iqra'uW
iqra'UA
iqra'na
\end{RLtext}

\medskip

\begin{RLtext}
\fullvocalize
in'a
in'ay
in'ayA
in'aW
in'aWA
in'ayna
\end{RLtext}

\medskip

\begin{RLtext}
u'mul

u'mulI

u'mulA
\end{RLtext}

\begin{RLtext}
iyqa.z

iyqa.zI

iyqa.zA
\end{RLtext}

\<u'mulI>
\<iyqa.zI>

\paragraph{Consistency of \protect\phon{hamza}}
with \exts{aa}, \exts{ii}, \exts{uu} needs to be revisited.

\medskip

\begin{RLtext}
iqra'I
iqra'ii
iqra'A
iqra'aa
iqra'UW
iqra'uuW
muqra'U
muqra'uu
\end{RLtext}

\medskip

\begin{RLtext}
ra''As ra''asa ru''isa mura''is
\end{RLtext}

\paragraph{Silent \protect\phon{taa' marbuu.taH} spelled as \protect\exts{H}}
needs to be revisited.%
\footnote{\url{http://ufal.mff.cuni.cz/~smrz/ICFP2006/icfp-encode.pdf\#page=10}}

\medskip

\begin{tabular}{l}
\<m_I'aHuN bi-al-mi'aHi>                \\
\<m_I'aH bi-al-mi'aT>                   \\
\<m_I'aTuN bi-al-mi'aTi>                \\
\<m_I'aHAni bi-al-mi'aTayni>            \\
\end{tabular}

\paragraph{Comparison with Arab\XeTeX}
can provide useful test cases for \phon{hamza}.%
\footnote{\url{http://mirrors.ctan.org/macros/xetex/latex/arabxetex/arabxetex.pdf\#page=6}}

\medskip

\begin{RLtext}
'amruN
'ibiluN
'u_htuN
'"u_ht"uN
'"Uql"Id"Is
ra'suN
'ar'asu
sa'ala
qara'a
bu'suN
'ab'usuN
ra'ufa
ru'asA'u
bi'ruN
'as'ilaTuN
ka'iba
qA'imuN
ri'AsaTuN
su'ila
samA'uN
barI'uN
sU'uN
bad'uN
^say'uN
^say'iN
^say'aN
sA'ala
mas'alaTuN
saw'aTuN
_ha.tI'aTuN
jA'a
ridA'uN
ridA'aN
jI'a
radI'iN
.daw'uN
qay'iN
.zim'aN
yatasA'alUna
'a`dA'akum
'a`dA'ikum
'a`dA'ukum
maqrU'aT
mU'ibAt
taw'am
yas'alu
'a.sdiq^A'uh_u
ya^g^I'u
s^U'ila
\end{RLtext}

\end{document}
